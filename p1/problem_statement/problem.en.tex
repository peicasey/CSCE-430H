\problemname{MM's Soccer Ranking System}

MM is a huge soccer fan and maintains a list of her favorite players. However, depending on a player's actions, positive or negative, she will want to demote or promote their ranking. She bases a player's new rank based off of how much she likes them relative to another player.

This can either be that she now likes a player X more than player B (player X's rank becomes player Y's rank), or that she likes player X less than player Y (player X's rank becomes player Y's rank - 1). The ranks below or above player X's new position will have to be bumped up or down depending on if X was moved up or down.

For example: MM's original list of players A--D are respectively ranked \#1--4. MM decides she now likes D better than Player A: that makes Player A's rank \#1 and Player B--D's rank \#2--4 (bumped down).

However, if MM decides if she likes A worse than D, then that makes A's new rank \#4 and Player B--D's rank \#1--3 (bumped up).

At the end she wants to see all the players and their new respective rankings.

\section*{Input}

The input consists of the following:

\begin{enumerate}
    \item The first line contains 2 integers: 
        \begin{enumerate}
            \item $n$ ($1 \leq n \leq 10^5$), the number of players in MM's ranking list.
            \item $k$ ($1 \leq k \leq 10^3$), the number of operations MM will run.
        \end{enumerate}
    \item The next $n$ lines contain a string representing the $player_i$'s name in the order of their original rank (from $1$st to $n+1$th).
    \item The next $k$ lines contain a sequence of operations. It will either be:
    \begin{enumerate}
        \item MM likes player X more than player Y now.
        \begin{lstlisting}
        player_X > player_Y
        \end{lstlisting}
        
        \item MM likes player X less than player Y now.
        \begin{lstlisting}
        player_X < player_Y
        \end{lstlisting}
    \end{enumerate}

    It is guaranteed that player X cannot be the same as player Y.
    
\end{enumerate}
    

\section*{Output}

At the end of all operations, output $n$ lines with each $ith$ line being:
\begin{lstlisting}
rank_i player_i
\end{lstlisting}

\section*{Sample Input 1}

\begin{verbatim}
4 3
Alejandro
Barry
Christiano
Don
Don > Alejandro
Barry < Don
Christiano < Alejandro
\end{verbatim}

\section*{Sample Output 1}

\begin{verbatim}
1 Don
2 Barry
3 Alejandro
4 Christiano
\end{verbatim}
